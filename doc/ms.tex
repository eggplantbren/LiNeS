\documentclass[a4paper, 12pt]{article}
\usepackage{amsmath}
\usepackage{dsfont}
\usepackage[utf8]{inputenc}
\usepackage{graphicx}
\usepackage[left=2cm, right=2cm, bottom=3cm, top=2cm]{geometry}
\usepackage{natbib}
\usepackage{microtype}

\title{LiNeS: Linked Nested Sampling}
\author{Brendon J. Brewer}
\date{}

\begin{document}
\maketitle

\abstract{\noindent Abstract}

% Need this after the abstract
\setlength{\parindent}{0pt}
\setlength{\parskip}{8pt}

\section{Introduction}
Some notation: $\pi(\theta)$ is the prior, $L(\theta)$ is the
likelihood function. The constrained prior corresponding
to level $j$ is
\begin{align}
p_j(\theta) &\propto
\frac{\pi(\theta)\mathds{1}\left[L(\theta) > L_j\right]}
{X_j}.
\end{align}

The LIS procedure, when applied to the NS sequence of
distributions, samples a probability distribution over
$\theta_{j,i}, K_j$ where $j$ indexes the level,
$i \in \{1, 2, ..., N\}$ is the
iteration within the level, and $K$ is a discrete
selection variable (one per level). The distribution for the
first level's quantities is
\begin{align}
p_{\rm LIS}\left(K_0, \left\{\theta_{0,i}\right\}\right)
&= \frac{1}{N} \prod_{i=K_0+1}^N T(\theta_{0,i} | \theta_{0,i-1})
\end{align}

\begin{thebibliography}{999}
\end{thebibliography}

\end{document}

